\documentclass[a4paper,10pt,oneside]{article}
%\usepackage[german]{babel}
\usepackage{calc}
\usepackage[latin1]{inputenc}
\usepackage{fancyheadings}
\usepackage{array}
\usepackage{epsfig}
\usepackage{float}
\usepackage{amsmath}
\usepackage{verbatim}
\usepackage{listings}

\rhead{}

\pagestyle{fancy}

%% SOME HELPERS %%%%%%%%%%%%%%%%%%%%%%%%%%%%%%%%%%

\newcommand{\entrylabel}[1]{\mbox{\textbf{#1}}}
\newenvironment{entry}[1]{
	\begin{list}{}{
		\renewcommand{\makelabel}{\entrylabel}
		\setlength{\labelwidth}{#1}
		\setlength{\leftmargin}{\labelwidth + \labelsep}
	}
}{\end{list}}

\begin{document}

%% �

%% COVER %%%%%%%%%%%%%%%%%%%%%%%%%1%%%%%%%%%%%%%%%%

\thispagestyle{empty}

\setlength{\parskip}{0mm}

\vspace*{\stretch{1}}
\begin{flushright}
\rule{\linewidth}{1mm}
	\Large{Projektpraktikum} \\
	\Huge{\textsf{Assembler and Disassembler in Gforth for ppc-\{32,64\}}} \\[8mm]
    \Large{Michal Revucky 0225176 534}
\rule{\linewidth}{1mm}
\end{flushright}


\vspace*{\stretch{4}}

\begin{center}

\Large \textsc{\today}
\end{center}

\pagebreak

%% TOC %%%%%%%%%%%%%%%%%%%%%%%%%%%%%%%%%%%%%%%%%%%%%%

\setcounter{page}{1}
\tableofcontents

\pagebreak

%% DOCUMENT %%%%%%%%%%%%%%%%%%%%%%%%%%%%%%%%%%%%%%%%%

% -) Manual fuer diesen assembler/disassembler:                                
%   welche instruktionen wie disassembliert werden, welche instruktion 
%   supported werden etc. 
% -) Approach:                                                                  
%     wie welche probleme geloest wurden, wieso und was ich dabei gelernt habe. 
% -) Testing:                                                                   
%     eine beschreibung der testscripte die ich geschrieben habe, sowie         
%     ausschnitte aus deren output.                                             
% -) Devel Env, sowie used tools, und ihre versionen.                           


%manual
\section{Manual}
\subsection{Postfix Notation}
    All words for this assembler/disassembler expect/return its arguments 
    as postfix operands. The best way to explain this is with an example.  
    When you open ~\cite{ppcman} at page 377, you see the syntax for the
    \texttt{add} instruction:
    
    \begin{verbatim}
        add rD,rA,rB
    \end{verbatim}
    This annotation is called infix it means that the instruction is
    followed by its operands.

    To translate this example into the postfix annotation, you simply
    need to put all the operands before the instruction. This would
    result in:

    \begin{verbatim}
        rD,rA,rB add
    \end{verbatim}
\subsection{Supported Instructions}
    This assembler/disassembler supports all instructions from ~\cite{ppcman}. 
    There is only one instruction which is only supported by the ppc32 and not 
    by the 64-bit implementation of this processor. Using a 64-bit instruction 
    on a ppc-32 will cause a run time error during the execution of your code,
    this also applies the other way round.

    Simplified mnemonics are not supported yet.

    The instruction of the m-form are disassembled differently than
    explained in ~\cite{ppcman}, when the \texttt{RC} bit is set, the dot is 
    placed at the first position of the instruction. The assembler accepts 
    both for example \texttt{rlwimi.} as well \texttt{.rlwimi}.

\subsection{Syntax}
    It is not necessary to explain the syntax of the disassembler in this
    section, because this section is intended to prevent the user from
    causing syntax errors, with the disassembler there is not much you can
    do wrong.

    All words provided by the \texttt{asm.fs} file which is usually located
    in \\ \texttt{/usr/share/gforth/<gforth-version>/arch/power} defines 
        words which take some arguments and depending on those they create a 
        32 bit long number which represents one instruction. The arguments 
        which take those words are integers only, whether signed or unsigned 
        depends on the instruction. The assembler has a boundary check and if 
        at least one argument is out of range it causes an error.

        It is also possible that you use hexadecimal numbers as well for 
        arguments, since the Gforth environment supports that feature:

        \begin{verbatim}1 2 15 add\end{verbatim}

        is equivalent to

        \begin{verbatim}$1 $2 $F add\end{verbatim}
        
        The arguments are converted by the Gforth run time internally.

        There are also instructions which take immediates as their arguments, 
        for example \texttt{addi}. Some assemblers/disassemblers support a 
        feature where they put for example a \texttt{r} before the number, so if
        somewhere in the input or output we have something like \texttt{r31}, we
        know that this is the register \texttt{31}. The point of this is that
        when using this assembler/disassembler you should know the order and the
        allowed ranges of the arguments pretty well, or have ~\cite{ppcman} 
        or similar by your hand.

    \subsection{Assembler}

        All the assembling words are defined in a word list called
        \texttt{assembler}. Usually Gforth does not look for words in that list,
        to activate this list you have to use the \texttt{code} word
        ~\cite{gforthman} page 157.

        The assembling words are made available either by calling:
            
            \begin{verbatim}
                gforth asm.fs
            \end{verbatim}

        where asm.fs is the source file which defines the assembling words, or
        when you compile Gforth and start it, the file will be loaded at 
        start up automatically.

        When you call \texttt{sc}, how it is defined and what it does i will try
        to explain with some code, the word \texttt{asm-sc}:

            \begin{verbatim}
                : asm-sc ( n "name" -- )
                  create ,
                does> @ 26 lshift 2 or h, ;
            \end{verbatim}

        is used to create the word \texttt{sc}, as you may noticed this word
        does not take any arguments. Now it is possible to create the word 
        \texttt{sc} by:

            \begin{verbatim}
                $11 asm-sc sc 
            \end{verbatim}

        When you invoke \texttt{sc} in a \texttt{code ... end-code} block, it
        fetches the number \texttt{\$11} which is the op-code for this
        instruction, makes a left shift by 26, so the op-code starts from bit 0
        and goes to bit 5, it sets bit 30 which is \texttt{2 or}, finally it 
        writes this machine code to \texttt{here}, which is performed by the 
        the word \texttt{h,}.
        
    \pagebreak
    \subsection{Disassembler}
        
        The disassembling words reside in word list called
        \texttt{disassembler}. To disassemble your favourite word you can do:
            \begin{verbatim}
            see +
            Code +  
            ( $10006DE0 ) 9 -32688 2 ld
            ( $10006DE4 ) 31 0 9 std
            ( $10006DE8 ) 0 8 26 ld
            ( $10006DEC ) 9 0 26 ld
            ( $10006DF0 ) 30 0 31 ld
            ( $10006DF4 ) 31 31 8 addi
            ( $10006DF8 ) 0 0 9 add
            ( $10006DFC ) 0 8 26 stdu
            ( $10006E00 ) 29 30 30 or
            ( $10006E04 ) 9 29 mtspr
            ( $10006E08 ) $10006120 b
            end-code
            \end{verbatim}
        The number in the parenthesis is the address of this instruction. If 
        \texttt{see} encounters a word which is provided in machine code, it 
        invokes \texttt{discode}. In \texttt{disasm.fs} the word 
        \texttt{discode} is redefined and its execution semantics is then same 
        as the one of \texttt{disasm}, it simply iterates over a block of
        memory, for each address and machine word it calls \texttt{disasm-inst}.
        This word fetches for every type of instruction the appropriate 
        disassemble xt and produces output like seen before.
        The output of the disassembler may be fed into the assembler.

\pagebreak 
%approach
\section{Approach-Implementation}

This chapter should provide an overview what problems arose while the
assembler/disassembler was written, it should also consider solutions for
certain problems which were chosen and the reasons why these were chosen.
As mentioned in ~\cite{gforthman} page 162, the proposal is to start with the 
disassembler and this is exactly what I did. I used the same approach as the 
author of the assembler/disassembler for the mips architecture.

    \subsection{Disassembler}

    \subsubsection{\label{cflags}Bit Fields}

    An instruction on the ppc-\{32,64\} is 32 bit long, there are several 
    instruction formats:

    \begin{itemize}
    \item A
    \item B
    \item D,DS
    \item I
    \item M,MD,MDS
    \item SC
    \item X,XFX,XFL,XL,XO,XS
    \end{itemize}

    Depending on the format there is certain information encoded in those 32
    bits. The op code always starts at bit 0 and goes till bit 5, so i introduced
    a convention how words are named which decode for example this 6 bit long
    number, in the range (0,5), this word is called \texttt{disasm-0,5}. The
    reason why I choose this convention is because for example with the
    \texttt{D} form, the manual ~\cite{ppcman} refers to the number in the 
    range (6,10), as \texttt{D} and for some other instructions of this form 
    it is \texttt{S}. I think that it is more readable/consistent to have my 
    convention.

    With bit or instruction fields I mean decoded numbers from a single 
    instruction, it may also be a single bit. The naming convention is
    (given in some pseudo BNF grammar):

    \begin{verbatim}
    disasm-range
    range :: number_start,number_end | number_end
    \end{verbatim}

    This scheme was used to implement all required instruction field decoding
    words, \texttt{31 - number\_end}, the result says how much the instruction
    should be right shifted, in most cases it was also required to mask this
    shifted number, so the result is not too long. When all required words were
    implemented, i had to make sure, that i did not make any typos, so I tested
    those words, but this will be discussed in the next chapter.

    \subsubsection{Flags}

    Some instructions have flags, those are their forms:

    \begin{itemize}
    \item A
    \item B
    \item I
    \item M MD MDS
    \item X XFL XL XS XO
    \end{itemize}

    I will explain this property which I call "flags" with the behavior of the
    \texttt{add} instruction. Depending on the \texttt{OE} and \texttt{Rc}
    flags, those are the
    bits 21 and 31 in the instruction, the mnemonic is different. The
    disassembling of this instruction is as following, first the arguments are
    disassembled then the mnemonic is displayed, in this example \texttt{add} is
    always displayed, since it is not affected by the \texttt{OE} or 
    \texttt{Rc} flags and in any case it is always the same. And then a number 
    is computed which is done by the word:

    \begin{verbatim}
    : disasm-xo-flags ( w -- u )
        dup disasm-21 1 lshift swap disasm-31 or ;
    \end{verbatim}

    It takes the instruction and the result tells us which additional string
    should be displayed, this number is passed directly to the word:

    \begin{verbatim}
    : get-xo,x,m,a-flag ( addr o -- addr )
        case
            0 of ." " endof
            1 of ." ." endof
            2 of ." o" endof
            3 of ." o." endof
       endcase ;
    \end{verbatim}

    When you take a look at the table \ref{xoflags} it is quite 
    intuitive what i did:

    \begin{table}
        \begin{center}
        \begin{tabular}{|r|rr|r|}
            \hline
            string & OE & Rc & Code\\
            \hline
            " " & 0 & 0 & 0 \\
            "." & 0 & 1 & 1 \\
            "o" & 1 & 0 & 2 \\
            "o." & 1 & 1 & 3 \\
            \hline
        \end{tabular}
        \end{center}
        \caption{\label{xoflags}Flags: example \texttt{xo} form}
    \end{table}

    So this number has a well-defined string which gives additional information
    and tells the programmer which flags are set.

    \subsubsection{Disasm Words}

    My work has the same logic as the \texttt{Mips} assembler/disassembler at
    this point I was inspired to implement several words like
    \texttt{define-format, disasm-table, disasm, disasm-inst}. 

    The \texttt{define-format} word is explained in ~\cite{gforthman}. I 
    started with the implementation of the disassembling words for the 
    \texttt{XO} form.  So I defined the two tables with \texttt{disasm-table}:

    \begin{verbatim}
        $40  disasm-table opc-tab-entry
        $200 disasm-table xo-tab-entry
    \end{verbatim}

    The \texttt{opc-tab-entry} table contains at a given position 
    a \texttt{xt}, this one either disassembles the instruction and displays
    the registers, immediates or it calls an \texttt{xt} which is an abstraction
    for instructions with extended op codes, finally it also may call
    \texttt{disasm-illegal}, which displays that an illegal instruction was
    encountered. The abstraction word was called \texttt{disasm-xo}:

    \begin{verbatim}
    : disasm-xo ( addr w -- )
        dup disasm-22,30 xo-tab-entry @ execute
        disasm-21,31 get-xo-flag drop ;
    ' disasm-xo 31 opc-tab-entry !
    \end{verbatim}

    Now i was able to do, \texttt{\$100117b0} is the address of the instruction
    and \texttt{\$7d8c1214} is the machine code:

    \begin{verbatim}
        $100117b0 $7d8c1214 disasm-inst
    \end{verbatim}

    This call outputs:

    \begin{verbatim}
        12 12 2 add ok
    \end{verbatim}

    This word (\texttt{disasm-inst}) first calls \texttt{disasm-0,5} in this 
    case it would be 31, so then it fetches the xt of \texttt{disasm-xo} from 
    the \texttt{opc-tab-entry} table and executes it. The next executed word
    from the \texttt{xo-tab-entry} in this case would be the word which
    extracts the registers from the instruction and displays it, it also
    displays the mnemonic of the instruction. Finally the flags are displayed
    by the \texttt{get-xo-flag} word.

    The next instruction form I wanted to implement at that time was the
    \texttt{X} form, this decision led me to a problem. The problem was that
    every instruction of this form has also \texttt{31} as its primary op code 
    which goes from bit \texttt{0} to \texttt{5}, it also has an extended 
    op code, but unlike the \texttt{XO} form the extended op code goes 
    from \texttt{21} to \texttt{31}. So I had to find some kind of an 
    abstraction which would either call \texttt{disasm-21,30} or 
    \texttt{disasm-22,30}. The word \texttt{disasm-31-similar} which does that 
    is:

    \pagebreak
    \begin{verbatim}
\ word which should be an abstraction for opcode 31 similar forms
: disasm-31-similar ( addr w -- )
    dup disasm-21,29 xs-tab-entry @ execute
    invert if disasm-31 get-xo,x,m,a-flag drop
        else dup disasm-21,30 x-31-tab-entry @ execute
            invert if disasm-31 get-xo,x,m,a-flag drop
            else dup disasm-22,30  xo-tab-entry @ execute
            invert if disasm-xo-flags get-xo,x,m,a-flag drop
                else disasm-illegal endif
            endif
        endif ;
' disasm-31-similar 31 opc-tab-entry !
    \end{verbatim}

    The principle of this word is that it uses tables which are initialized
    differently than for example that of the \texttt{opc-tab-entry} table. If
    the default behaviour of an execution token at a certain position is not
    overridden by the means of entering a disassembling token for a certain
    code, it just pushes \texttt{true} on the stack by calling 
    \texttt{disasm-unknown}. Let's demonstrate this with an example, we want 
    to disassemble an instruction of the form \texttt{X}:

    \begin{enumerate}
    \item It has opc \texttt{31} so \texttt{disasm-31-similar} is called from
    \texttt{opc-tab-entry}.
    \item The \texttt{disasm-31-similar} checks in the next step if a
    disassemble word is available from the \texttt{xs-tab-entry} table. In 
    this case it's not available so the default behavior pushes 
    \texttt{true} on the stack.
    \item The first else part is executed but now a disassemble word in the 
    \\\texttt{x-31-tab-entry} is being found, so it displays the arguments and 
    the mnemonic of the instruction, it also leaves \texttt{false} on the stack,
    which causes next line to be executed. The remaining part is not executed
    because of \texttt{disasm-31-similar}'s semantic.
    \end{enumerate}

    Finally I had to create words which would be an abstraction for all 
    available forms:

    The word
    \texttt{disasm} behaves differently on a ppc32 than on ppc64, for a ppc32
    for each address it fetches the instruction, which is 32 long and calls
    disasm-inst. For a ppc64 it is a bit trickier, when fetching from an 
    address, the result is 64 bit long, the instruction is placed in the upper 
    32 bits, some shifting has to be done in order to get the 32 bit long
    instruction, this is done by the \texttt{get-inst} word and the result is
    passed to the \texttt{disasm-inst} word.
    The definition of the \texttt{disasm} word is placed in two
    \texttt{[if] ...  [endif]} blocks and depending on the size of one cell
    either one or the other is being used:

    \pagebreak
    \begin{verbatim}
    \ for ppc32
    1 cells 4 =
    [if]
    : disasm ( addr u -- )
        bounds u+do
            cr ." ( " i hex. ." ) " i i @ disasm-inst
            1 cells +loop
        cr ;
    [endif]

    \ for ppc64
    1 cells 8 =
    [if]
    : get-inst ( u -- o )
        32 rshift $FFFFFFFF and ;

    : disasm ( addr u -- )
        bounds u+do
            cr ." ( " i hex. ." ) " i i @ get-inst disasm-inst
            \ next inst plus 4
            4 +loop
        cr ;
    [endif]
    \end{verbatim}

    The words for the remaining forms are implemented in some manner as
    described above. I followed for each form one of the appropriate scheme as
    described above and finished the disassembler.

    \subsection{Assembler}

    \subsubsection{Utils-Helpers}

    First I had to implement some helper words:

    \begin{enumerate}
        \item \texttt{h,}:
        
        This word behaves differently on a ppc32 than on a ppc64,
        on a ppc32 it simply calls \texttt{,}. On a ppc64 if either an aligned
        address is available then the machine code has to be shifted left by 32,
        then it is stored to \texttt{here}, or otherwise the machine code is
        stored to \texttt{here - 4}, finally 4 is allotted in both cases.

        \pagebreak
        This is the code of the word \texttt{h,} for \texttt{ppc64}

        \begin{verbatim}
        : h, ( h -- ) 
            here here aligned = if
                32 lshift
                here !
            else
                here 4 - dup
                @ rot or
                swap !
            endif
            4 allot ;
        \end{verbatim}

        \item \texttt{check-range}:

        This word is used every time when numerical arguments are being
        processed and it has to be ensured that these are within a certain
        range. For example registers can only in the range from \texttt{0} to
        \texttt{31} so whenever an argument is to be processed which represents
        a register this word is used like this:

        \begin{verbatim}
        <value_to_check> 0 $20 check-range
        \end{verbatim}

        If the \texttt{<value\_to\_check>} is out of range, an 
        "illegal numerical argument" exception is risen.

        \item \texttt{concat}:

        This word is often used by the \textit{top-level} defining words, those
        are used to define the assembling words. \texttt{Concat} is used
        whenever a word for a form is defined which contains so called
        \textit{flags} as described in section \ref{cflags}.

        What it does is, it takes two strings in the \texttt{address, length}
        form and concats those two into a single one. Since string handling in
        Forth is not that intuitive, there has to be allocated space at the end
        of the string where the second string should be appended to. By the
        definition of these word:

        \begin{verbatim}
        : copy>here ( a1 u1 -- )
            chars here over allot swap cmove ;

        : concat ( a1 u1 a2 u2 -- a u )
            here >r 2swap copy>here copy>here r>
            here over - 1 chars / ;
        \end{verbatim}

        We can demonstrate how it works:

        \begin{verbatim}
        s" foo" s" bar" concat .s <2> -1213708974 6  ok
        type foobar ok
        \end{verbatim}

    \end{enumerate}

    \subsubsection{Top Level Defining Words}

    Every defining word is using the \texttt{create does>} construct, the
    \texttt{does>} part builds one machine word from right to left by consuming
    the arguments which are given to the assembling word and sets the
    appropriate bits.
    I will explain the principle with some examples, these words can be divided
    into two major categories:

    \begin{enumerate}
    \item Words which takes only one number, which is either the primary
    op code or the extended op code, if it is the extended op code, then the
    primary op code is hard coded in the \texttt{create} part. 
    The \texttt{asm-sc} word from previous chapter is of this kind.

    \item Words which are meta defining, they are used for all forms which
    use \textit{flags} as explained in the \ref{cflags} section, this type 
    will be discussed extensive in the rest of this section.
    \end{enumerate}

    The problem with the assembler and the \texttt{flags}, section
    ~\ref{cflags}, was that I had to factor the flag setting semantic. For 
    example with the \texttt{X} form four different defining words would 
    be required, because there are four different possibilities how the bit 
    flags may be set.  With my approach the setting of the flags bits is done 
    by extra words.

    The meta definition word would be:

    \begin{verbatim}
    \ name as addr length
    : asm-xo-1-define ( xt n "name" "name" -- ) 
        concat nextname
        create 2, 31 ,
    does> dup dup @ swap 1 cells + @ execute ( D A B -- )
        asm-1-16,20 asm-1-11,15 asm-1-6,10 asm-1-0,5 h, ;
    \end{verbatim}

    As you can see here we use the \texttt{concat} word, which was explained
    in the previous section. This word takes an \texttt{xt}, this one sets the
    bit flags. It also takes a number which in this case is the extended op code.

    The \texttt{does>} part first fetches the \texttt{xt} and executes it, by
    executing it, the flag bits are set, this word also sets the bits in range
    of the extended op code. The extended op code is fetched before the 
    execution of this word. As a comment you can see \texttt{( D A B -- )}, 
    those are the arguments for the assembling words of type \texttt{XO}, which 
    take three registers. The words \texttt{asm-1-16,20}, \texttt{asm-1-11,15} 
    and \texttt{asm-1-6,10} consume and set the bits in appropriate range, the 
    range is also mentioned in the name of the word. The word 
    \texttt{asm-1-0,5} sets the primary op code, this one fetched from memory.
    
    The definition \texttt{asm-xo-1} word which uses the  
    meta definition word \\\texttt{asm-xo-1-define} is:

    \begin{verbatim}
    : asm-xo-1 ( n "name" -- )
        name { n addr len }
        ['] asm-xo-21,31-00 n addr len s" " asm-xo-1-define
        ['] asm-xo-21,31-01 n addr len s" ." asm-xo-1-define
        ['] asm-xo-21,31-10 n addr len s" o" asm-xo-1-define
        ['] asm-xo-21,31-11 n addr len s" o." asm-xo-1-define  ;
    \end{verbatim}

    By the line \texttt{\$10A asm-xo-1 add} we create in this case four
    different words \texttt{add}, \texttt{add.}, \texttt{addo} and
    \texttt{addo.}. With this approach we do not require so much code as with 
    a naive approach.


\pagebreak
%testing
\section{Testing}

\subsection{Bit Fields}

The bit fields words were described in the previous chapter, when I finished the
implementation of them, I wrote a simple shell script (listing \ref{sbitfiels})
which made sure that I discovered all my typos:

\begin{lstlisting}[float, caption=Bit fields test script, label=sbitfiels]
#!/bin/sh
# script: inst_field_test.sh
# purpose: test disasm-<number> and 
# disasm-<number>,<number> words

gforth='~/gforth-20050128-ppc64/bin/gforth-fast'
disasm='~/praktikum/ppc/disasm.fs'

# ranges
echo "testing single:";
echo "---------------";
for k in $(cat befehle_binaer);
 do
   for l in $(cat to_test);
    do
      first=`echo $l | cut -d, -f1`
      second=`echo $l | cut -d, -f2`
      len=`echo $second - $first + 1 | bc`
      echo -n "$k disasm-$first,$second: ";
      $gforth $disasm -e "%$k disasm-$first,$second 
        dup hex.  %${k:$first:$len} dup hex. = . bye"
      echo ""
    done
done

# single bits
echo "testing single bits:";
echo "--------------------";
for k in $(cat befehle_binaer);
do
    for l in $(cat to_test2);
    do
        bit=`echo $l`
        echo -n "$k disasm-$bit: ";
        $gforth $disasm -e "%$k disasm-$bit 
            dup hex. %${k:$bit:1} dup %hex. = . bye"
        echo ""
    done
done
\end{lstlisting}

First I defined two global variables which are used to load the
\texttt{disasm.fs} source file and to identify the path to the 
the Gforth binary. Two files \texttt{to\_test} and \texttt{to\_test2} are used
to identify the number(s) which are to be inserted behind \texttt{disasm-}.
In a file called \texttt{befehle\_binaer}, I created some random strings which
have the length 32, because an instruction is 32 bit long (befehle is german for
instructions and binaer for binary) and they represent a single instruction. 
Since those strings are binary they are either composed of \texttt{0} or 
\texttt{1}. Those files may be located in any directory but as this script is
implemented currently, every file has to be in the same directory. This script
works only if every \texttt{disasm-<n1>[,<n2>]} words are defined in gforth's
standard word list.

In the first block, words of the form \texttt{disasm-<n1>,<n2>} are
tested. For every in line the \texttt{befehle\_binaer} file every possible 
\texttt{disasm-<n1>,<n2>} word is tested. In the second nested loop the most 
interesting lines are:

\begin{verbatim}
echo -n "$k disasm-$first,$second: ";
$gforth $disasm -e "%$k disasm-$first,$second 
    dup hex. %${k:$first:$len} dup hex. = . bye"
\end{verbatim}

The first line displays the instruction encoded in binary and the
\texttt{disasm} word which is applied. In the second line the output
of \texttt{disasm-<n1>,<n2>} is compared with a reference. The comparison is
done with the \texttt{=} Gforth word. If the result is same with the reference
true should be displayed which is \texttt{-1} in Gforth. For debugging purposes
the reference and the result is displayed. The \texttt{\$\{k:\$first:\$len\}}
shell command extracts the reference from \$k which represents the whole
instruction. A string which starts at \$first and is \$len long is extracted.
The block which is used to test words of type \texttt{disasm-<n>} is quite the
same with the exception that it used the \texttt{to\_test2} file. It was easy to
identify failed test cases, every line which had a \texttt{0} at the end, was a
\texttt{FAILED} test case.

When we call this script it produces output like listed in listing \ref{sbitout}
(actually it is much more output than this).

\begin{lstlisting}[float, language=csh, caption=Bit field test output,
label=sbitout]
testing single:
---------------
10001110111100110100010110100100 disasm-0,5: $23 $23 -1 
10100001101000101011001111000011 disasm-0,5: $28 $28 -1 
11010001101000101011001111000011 disasm-0,5: $34 $34 -1 
11110101010000111110110000101010 disasm-0,5: $3D $3D -1
.
.
.

testing single bits:
--------------------
10001110111100110100010110100100 disasm-31: $0 $0 -1 
11110001001000110100010101101011 disasm-31: $1 $1 -1 
.
.
.
\end{lstlisting}

\subsection{Test Structure}

In order to test the assembler/disassembler I had to create a testing structure
where I had all my test cases. I created this structure:

\begin{verbatim}
micrev@north:~/praktikum/test/mnemonic$ ls
a   find_mnemonic.sh  make_unique.py test_all_forms.sh    xfl  xs     
b   find_mnemonics.sh md             test_asm.py          xfx 
d   i                 mds            test_disasm-inst.py  xl
ds  m                 sc             x                    xo
\end{verbatim}

In this listing, all names which are the same as the forms of the
\texttt{ppc-\{32,64\}} architecture are directories.
Those directories contain mainly text files with instructions, those
were generated using \texttt{objdump} and 3 scripts.

The script in listing \ref{smnem} was used in order to search through all
binaries available at a machine for a particular mnemonic of a particular form:

\begin{lstlisting}[float, caption=Script: find\_mnemonic.sh, label=smnem]
#!/bin/sh

# Autor Michal Revucky 
# Purpose: find a certain mnemonic specified by $1 and 
#          write to a file  it writes only a specified 
#          number of lines into to result file which 
#          is determined by $COUNT, the file name 
#          contains the mnemonic too. it checks all 
#          binaries from DIRS, it places the result file 
#          into ./<form>/$HOSTNAME.$1
#          Usage: ./find_mnemonic <mnemonic> <form>

DIRS="/home/complang/micrev/gforth-20050128-ppc64/bin 
/bin /usr/bin /usr/local/bin /usr/X11R6/bin 
/usr/powerpc64-unknown-linux-gnu/gcc-bin/3.4.3 
/opt/Ice-2.0.0/bin"

COUNT=100

if [ $# -ne 2 ]; then 
    echo "usage: $0 <mnemonic> <form>"
    exit
fi

if [ ! -d $2 ]; then 
    mkdir $2
fi

for k in $DIRS; do
 for l in `ls $k`; do
  objdump -d $k/$l | grep $1 
    | head -n1 >> $2/$HOSTNAME.$1
   if [ `wc -l $2/$HOSTNAME.$1 
      | grep -o [0-9]*` -ge $COUNT 
      ] ; 
   then exit
   fi
  done
done
\end{lstlisting}

For each Form I created a file called \texttt{mnemonics} in other words: it was 
in every directory as mentioned above and contained every mnemonic for a 
particular form. So I wrote another script \texttt{find\_mnemonics.sh} 
(listing \ref{smnems}) which would be faster, this time you had only to 
specify the form and it did exactly the same as the script 
\texttt{find\_mnemonic.sh}, but for every mnemonic of the specified form.

\begin{lstlisting}[float, caption=Script: find\_mnemonics.sh, label=smnems]
#!/bin/sh

# Autor Michal Revucky
# Purpose: find certain mnemonics of form specified 
#          by $1 and write to a file it writes only
#          a specified number of lines into to result
#          file which is determined by $COUNT, the 
#          file name contains the mnemonic too. it 
#          checks all binaries from DIRS, it places 
#          the result file into ./<form>/$HOSTNAME.$1
#          it takes every mnemonic from $1/mnemonic
# Usage: ./find_mnemonic <form>

DIRS="/home/complang/micrev/gforth-20050128-ppc64/bin 
/bin /usr/bin /usr/local/bin /opt/Ice-2.0.0/bin
/usr/powerpc64-unknown-linux-gnu/gcc-bin/3.4.3 
/usr/X11R6/bin"

COUNT=100

if [ $# -ne 1 ]; then 
    echo "usage: $0 <form>"
    exit
fi

if [ ! -d $1 ]; then 
    mkdir $1
fi

for j in `cat $1/mnemonics`; do
 echo "checking $j"
 touch $1/$HOSTNAME.$j;
  for k in $DIRS; do
    for l in `ls $k`; do
      if [ `wc -l $1/$HOSTNAME.$j 
         | grep -o [0-9]*` -lt $COUNT 
         ] ; then 
         objdump -d $k/$l | grep $j 
            | head -n1 >> $1/$HOSTNAME.$j ;
      fi
    done
  done
done
\end{lstlisting}

The files which contain the test cases have a form as shown in listing 
\ref{stfile}.

\begin{lstlisting}[float, caption=Example of a test file, label=stfile]
    100117b0:   7d 8c 12 14     add     r12,r12,r2
    1001132c:   7c 1c ba 14     add     r0,r28,r23
    100112e8:   7c 00 52 14     add     r0,r0,r10
    .
    .
    .
\end{lstlisting}

After the execution of \texttt{find\_mnemonics.sh} I checked the files which
contained the instructions and what I found was that their content was not
unique i.e. \texttt{add r0,r0,r1} appeared in the file more the once. From the
point of view of testing it would not matter, but I wanted the files unique, so
I wrote a Python (listing \ref{suni}) script:

\begin{lstlisting}[float, language=python, caption=Script: make\_unique.py, 
label=suni]
#!/usr/bin/env python

# makes testfiles, form: <hostname>.<mnemonic> unique
# $1 specifies the form to make unique

import commands
from optparse import OptionParser

HOSTNAME = commands.getoutput("hostname")
MNEMONIC_TEST='~/praktikum/test/mnemonic'

def get_opc_code_hex(line) :
    l = line.split()
    return l[1]+l[2]+l[3]+l[4]

    if __name__ == '__main__' :
        parser = OptionParser("%prog <form>")
        (options, args) = parser.parse_args()
        if len(args) < 1 or len(args) > 1 :
            parser.error("incorrect number of arguments")
        dir = MNEMONIC_TEST + '/' + args[0]
        for k in commands.getoutput(
                'ls %s/%s*'
                %(dir, HOSTNAME)).split('\n') :
            print k
            f = open(k,'r')
            lines = f.readlines()
            new_file = open(k+'.unique', 'w') 
            tmp_list = []
            new_file_list = []
            for l in lines :
                opc_code_hex = get_opc_code_hex(l)
                if opc_code_hex not in tmp_list :
                    tmp_list.append(opc_code_hex)
                    new_file_list.append(l)
            new_file.writelines(new_file_list)
            new_file.close()
            print commands.getoutput('mv 
                %s.unique %s' %(k,k))
\end{lstlisting}

For all test files of the form \texttt{<hostname>.<mnenonic>} this script 
iterates over every single line of any given test file of a particular form. 
For each line the machine code in the hexadecimal form is parsed from the file,
a list holds all hexadecimal numbers which are already used,
only instructions which are not already in this list are written to the new
test file which is called \\\texttt{<hostname>.<mnemonic>.unique}, finally this
file is renamed to its final name.

\subsection{Mnemonic Tests}

When I set up the directories with the test cases I had to write some scripts
which would parse the instructions from the test files, call \texttt{gforth} and
compare the output with a reference. For the disassembler I wrote the script
called \texttt{test\_disasm-inst.py} and \texttt{test\_asm.py} for the 
assembler . I decided to use Python for this, because it has a lot of useful 
libraries i.e. regular expression for parsing.

\subsubsection{Disassembler}

The usage of the \texttt{test\_disasm-inst.py} is:

\begin{verbatim}
$ ./test_disasm-inst.py -h
usage: test_disasm-inst.py [-m] [-a] form [mnemonic]

options:
-h, --help      show this help message and exit
-m, --mnemonic  compare the returned mnemonic
-a, --args      compare the result args
\end{verbatim}

As you can see, you have to specify a form to be tested, if you want to test a
particular mnemonic only, you can specify an optional mnemonic. The options are
not really useful, because I always called the script with \texttt{-m} and
\texttt{-a}. When the script was called, then for each test case in a directory
it iterated over each line of a test file and did this:

\begin{enumerate}

    \item parsed a line which had the form as mentioned in listing \ref{stfile}
    and created a Python tuple which is easier to use in a Python script.
    \item call Gforth with arguments which ware taken from the tuple
    created in the previous step.
    \item compare the results returned by Gforth and display a status either
    "OK" or "FAILED".

\end{enumerate}

The output in listing \ref{sdout} is produced by the 
\texttt{test\_disasm-inst.py} script, the first tuple (followed by "testing") 
is the information which is parsed from the test 
files. The first element is the address of the instruction, followed by the 
instruction in hex, then by the mnemonic and finally by the arguments of the 
instruction wrapped in a python list. Either "OK" or "FAILED" denotes the status
of this particular instruction which is tested. The tuple which is followed by
"disasm-inst" contains elements which is returned by the \texttt{disasm-inst}
word. The first element is the returned mnemonic followed by the arguments which
were disassembled wrapped in a Python list. These two elements are parsed from
the third one, which is Gforth's output. As you can see the second and third
element of the first tuple is compared with the first and second element of the
second tuple, if they match this part of a test case succeeds. When you run this
with \texttt{-m} and \texttt{-a} option it also displays a number of failed
test cases.

\begin{lstlisting}[float, caption=Output of test-disasm-inst.py, label=sdout]
Testing mnemonic and its arguments:
whole form from ~/praktikum/test/mnemonic/md
=========================================================
testing: ( '100022d0' , '782106e9'
         , 'rldic.'
         , ['1', '1', '0', '59']
         ) OK
disasm-inst: ( 'rldic.'
             , ['1', '1', '0', '59']
             , '1 1 0 59 rldic.'
             )
---------------------------------------------------------
testing: ( '100012c0'
         , '782106e8'
         , 'rldic'
         , ['1', '1', '0', '59']
         ) OK
disasm-inst: ( 'rldic'
             , ['1', '1', '0', '59']
             , '1 1 0 59 rldic'
             )
---------------------------------------------------------
\end{lstlisting}

\subsubsection{Assembler}

The script \texttt{test\_asm.py} is quite the same as the script
\texttt{test\_disasm-inst.py}. Since I wrote this script after the disassembler
script I knew that I would not need a bunch of options so the usage is quite
simple:

\begin{verbatim}
$ ./test_asm.py -h
usage: test_asm.py <form>

options:
  -h, --help  show this help message and exit
\end{verbatim}

This time the script works the other way:

\begin{enumerate}
    \item parse the processed line and create the tuple followed by "testing",
    as you may see in the output of \texttt{test\_asm-inst.py} script
    (listing \ref{saout}), the elements are the same as in listing \ref{sdout}.
    \item call Gforth, the first element of the second tuple from listing
    \ref{saout}, is the command Gforth is called with. The second element in 
    the second tuple is the result and it is compared with the second element 
    of the first tuple.
    \item if the result matches the reference, "OK" is displayed, the script
    continues.
\end{enumerate}

In order to use the script the way it is currently implemented you should
comment the words which make the assembling words appear in the word list
\texttt{assembler} and uncomment the additional output in the \texttt{h,} word
which causes to output the machine code of each instruction which is being
assembled.

\begin{lstlisting}[float, caption=Output of test\_asm.py, label=saout]
Testing mnemonic and its arguments:
whole form from ~/praktikum/test/mnemonic/md
=========================================================
testing: ( '100022d0', '782106e9'
         , 'rldic.'
         , ['1', '1', '0', '59']
         ) OK
asm-inst: (' 1 1 0 59 rldic.', '$782106E9')
---------------------------------------------------------
testing: ( '100012c0'
         , '782106e8'
         , 'rldic'
         , ['1', '1', '0', '59']
         ) OK
asm-inst: (' 1 1 0 59 rldic', '$782106E8')
---------------------------------------------------------
\end{lstlisting}

I also created a simple wrapper shell script (listing \ref{sallforms}) which 
calls the two mentioned python scripts and displays only "interesting" stuff.

\begin{lstlisting}[caption=Script: test\_all\_forms.sh, label=sallforms]
#!/bin/sh

FORMS="a b d ds i m md mds sc x xfl xfx xl xo xs"

echo "disassembler"
for k in  $FORMS; do
    ./test_disasm-inst.py -m -a $k 
        | egrep 'form|Testcases' ;
    echo "==============="
done

echo "assembler"
for k in  $FORMS; do
    ./test_asm.py $k | egrep 'form|Testcases' ;
    echo "==============="
done
\end{lstlisting}

\pagebreak
%tools used
\section{Development Environment}

The Assembler/Disassembler was developed in this environment:

\begin{itemize}
    \item Gforth version 0.6.2-20030910
    \item The processor G4 was used for the 32 bit version.
    \item The processor G5 was used for the 64 bit version.
    \item Python/Shell script for testing.
    \item Subversion for versioning and backup, the repository 
    was checkout on several machines
    \item The source war written using the Vim editor.
    \item \LaTeX was used to write this paper.
\end{itemize}

\pagebreak
\begin{thebibliography}{4}
\bibitem[PPC-MANUAL]{ppcman} PowerPC Microprocessor Family Programming
Environments Manual for 64 and 32 Bit Microprocessors, Version 2.0
\bibitem[Gforth-MANUAL]{gforthman} Gforth Manual for version 0.6.2 Neal Cook,
Anton Ertl, David Kuehling, Bernd Paysan, Jens Wilke
\end{thebibliography}

% \appendix
% appendix additional or some other stuff
\end{document}
