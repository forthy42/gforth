\section{Manual}
\subsection{Postfix Notation}
    All words for this assembler/disassembler expect/return its arguments 
    as postfix operands. The best way to explain this is with an example.  
    When you open ~\cite{ppcman} at page 377, you see the syntax for the
    \texttt{add} instruction:
    
    \begin{verbatim}
        add rD,rA,rB
    \end{verbatim}
    This annotation is called infix it means that the instruction is
    followed by its operands.

    To translate this example into the postfix annotation, you simply
    need to put all the operands before the instruction. This would
    result in:

    \begin{verbatim}
        rD,rA,rB add
    \end{verbatim}
\subsection{Supported Instructions}
    This assembler/disassembler supports all instructions from ~\cite{ppcman}. 
    There is only one instruction which is only supported by the ppc32 and not 
    by the 64-bit implementation of this processor. Using a 64-bit instruction 
    on a ppc-32 will cause a run time error during the execution of your code,
    this also applies the other way round.

    Simplified mnemonics are not supported yet.

    The instruction of the m-form are disassembled differently than
    explained in ~\cite{ppcman}, when the \texttt{RC} bit is set, the dot is 
    placed at the first position of the instruction. The assembler accepts 
    both for example \texttt{rlwimi.} as well \texttt{.rlwimi}.

\subsection{Syntax}
    It is not necessary to explain the syntax of the disassembler in this
    section, because this section is intended to prevent the user from
    causing syntax errors, with the disassembler there is not much you can
    do wrong.

    All words provided by the \texttt{asm.fs} file which is usually located
    in \\ \texttt{/usr/share/gforth/<gforth-version>/arch/power} defines 
        words which take some arguments and depending on those they create a 
        32 bit long number which represents one instruction. The arguments 
        which take those words are integers only, whether signed or unsigned 
        depends on the instruction. The assembler has a boundary check and if 
        at least one argument is out of range it causes an error.

        It is also possible that you use hexadecimal numbers as well for 
        arguments, since the Gforth environment supports that feature:

        \begin{verbatim}1 2 15 add\end{verbatim}

        is equivalent to

        \begin{verbatim}$1 $2 $F add\end{verbatim}
        
        The arguments are converted by the Gforth run time internally.

        There are also instructions which take immediates as their arguments, 
        for example \texttt{addi}. Some assemblers/disassemblers support a 
        feature where they put for example a \texttt{r} before the number, so if
        somewhere in the input or output we have something like \texttt{r31}, we
        know that this is the register \texttt{31}. The point of this is that
        when using this assembler/disassembler you should know the order and the
        allowed ranges of the arguments pretty well, or have ~\cite{ppcman} 
        or similar by your hand.

    \subsection{Assembler}

        All the assembling words are defined in a word list called
        \texttt{assembler}. Usually Gforth does not look for words in that list,
        to activate this list you have to use the \texttt{code} word
        ~\cite{gforthman} page 157.

        The assembling words are made available either by calling:
            
            \begin{verbatim}
                gforth asm.fs
            \end{verbatim}

        where asm.fs is the source file which defines the assembling words, or
        when you compile Gforth and start it, the file will be loaded at 
        start up automatically.

        When you call \texttt{sc}, how it is defined and what it does i will try
        to explain with some code, the word \texttt{asm-sc}:

            \begin{verbatim}
                : asm-sc ( n "name" -- )
                  create ,
                does> @ 26 lshift 2 or h, ;
            \end{verbatim}

        is used to create the word \texttt{sc}, as you may noticed this word
        does not take any arguments. Now it is possible to create the word 
        \texttt{sc} by:

            \begin{verbatim}
                $11 asm-sc sc 
            \end{verbatim}

        When you invoke \texttt{sc} in a \texttt{code ... end-code} block, it
        fetches the number \texttt{\$11} which is the op-code for this
        instruction, makes a left shift by 26, so the op-code starts from bit 0
        and goes to bit 5, it sets bit 30 which is \texttt{2 or}, finally it 
        writes this machine code to \texttt{here}, which is performed by the 
        the word \texttt{h,}.
        
    \pagebreak
    \subsection{Disassembler}
        
        The disassembling words reside in word list called
        \texttt{disassembler}. To disassemble your favourite word you can do:
            \begin{verbatim}
            see +
            Code +  
            ( $10006DE0 ) 9 -32688 2 ld
            ( $10006DE4 ) 31 0 9 std
            ( $10006DE8 ) 0 8 26 ld
            ( $10006DEC ) 9 0 26 ld
            ( $10006DF0 ) 30 0 31 ld
            ( $10006DF4 ) 31 31 8 addi
            ( $10006DF8 ) 0 0 9 add
            ( $10006DFC ) 0 8 26 stdu
            ( $10006E00 ) 29 30 30 or
            ( $10006E04 ) 9 29 mtspr
            ( $10006E08 ) $10006120 b
            end-code
            \end{verbatim}
        The number in the parenthesis is the address of this instruction. If 
        \texttt{see} encounters a word which is provided in machine code, it 
        invokes \texttt{discode}. In \texttt{disasm.fs} the word 
        \texttt{discode} is redefined and its execution semantics is then same 
        as the one of \texttt{disasm}, it simply iterates over a block of
        memory, for each address and machine word it calls \texttt{disasm-inst}.
        This word fetches for every type of instruction the appropriate 
        disassemble xt and produces output like seen before.
        The output of the disassembler may be fed into the assembler.
